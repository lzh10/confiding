\section{六月四日}
\vspace*{2\ccwd}
\begin{adjustwidth}{0cm}{0cm}
    \Large\ckai{
        王: 

        \hspace*{2em} 向你道一声好。已是多久未抒发了,这淤积的浩漫情绪啊,壅阻在了喉口,请与我决崩,好让万方的积漩舒然奔畅。   

        \hspace*{2em} 时维孟夏,深夜的风里含着温柔,这渴念中的情人才能予的温柔,更令我无措,心中强烈的想像,直想到了绝处,
        我还能拥有什么,我还能守盼什么,所有欢长并非无终竟,已后当如何。
        
        \hspace*{2em} 偏让我感知许多的欢喜悲愁,又不允自抉,生命其无理也极,造物真个善作弄人。便作了木偶亦未可悲也,可悲是明晓了
        己之木偶身,仍作木偶事,不似木偶心。贪了不可承之欲,得尝其苦,冤理都无从说。而今,便只可向你絮絮牢骚,月儿也避了我的灼灼目光
        掩向云层。念天地之悠阔,纳不下一个自我缩蜷的灵魂,岂不能狭间里游刃?

        \hspace*{2em} 情绪不觉消散了大多,是世情对我同化深至。近来已无新得的感想可以表诉,重行往复,岁日徒增,希念余烁也不若萤火,
        存生,止行口食之道也已,何意旁加使命。是故,我之欲求皆付枕中,梦醒,尽落纷纷。

        \hspace*{2em} 劳事伤损,灵泉近枯,心意再莫能与幻渺的时空勾连,便是再无托愿,便是再无归念,独孑一身于迷丛里失路,似也年少玩戏
        摆弄的蚁虫,向时的命运,今主易换了客,非枉也。如是命定,如是兀兀以卒年。

        \hspace*{2em} 闲定而想,突然地觉知,竟无可为,竟无所愿为,是心力空乏的绝望。啊\~{}心中无以言词的曲转,无贴词的歌,以和应的唱调、
        以谐韵的律奏,只在脑海作无声的绕旋,只作无容的悲戚,已而终日恍恍焉。

        \hspace*{2em} 千万端的感念,当如何成言,今已失却了巧辞,我的拙舌只作得复赘,望你莫见嫌于我祥林嫂般的絮言。我虽愿向你直陈所有
        心内的隐怀,然惯于自语,其言闪烁,含义晦明,对你,须得另作分说,却又分说不得,我竟又不明你是哪个了。我赋予你一部分的我,一部分
        暗里倾心的对象,一部分旁视的第三人,余下,我盼你是个长者智者:愿你聆我、爱我、知我、解我,或是策我、醒我、分剖我、醍醐我,因我的
        心神已毫无主意,哪向亮有一盏薄明的光?
        
        \hspace*{2em} 穿悬天穹灿明的星河啊,愿有我的终宿。一引浮槎纵去兮,邈邈不归意。

}
\end{adjustwidth}
\newpage