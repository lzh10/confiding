\section{四月十六}
\vspace*{2\ccwd}
\begin{adjustwidth}{0cm}{0cm}
    \Large\ckai{
        王: 

        \hspace*{2em} 这是我第一次给你写信,不确知你是否姓王,或真否有你这么一个人。借“王”来称你
        是本着敝姓“李”,此外“王”有更高概率可能作你的姓。且不管此无紧要的事,需再论你的性别,依我所想:
        若你为男性,向你吐诉情感的细腻心思太显矫情;若你为女性,倘使大谈对另些女子的倾慕,又该怎样的
        不识趣。 故而让我脱越从性别有可引发的愚见,只作两个意识体的交互。今次从简单的认识始,不涉具象
        的话题。

        \hspace*{2em} 我呀,实非一个爽落的人,在之后的交往中你会有真切地感受。源自根深的不自信,那种
        从顶至踵、由内及外的可称全面的自我否定,频繁而又下意识在样貌与智识上进行自我打击。如此确信
        我是这样一个微末的人:立污灰之上尚无以显亮,伴尘埃而列亦不能出兀。所以,你莫要讶异今后我在
        种种强烈的渴盼之后仍难抉择而退舍的怯弱。自明了自己的这种性格后,常常陷入忧烦中,这忧烦随
        时日生长渐渐占有了我所余全部空暇的思想,当我忙完生计和偶尔的兴致后,在平静的环境中放松心神,
        就会突然的有一股悸动,继而类似绝望的悲绪那么一点点地蔓延开来。你能明白这种感受么,像是内里
        毫无预兆地被攻破,侵入者肆无忌惮的劫掠美好的留存。啊啊\~{}这些本已无多的遗珍至宝。我常常是这样的
        自哀,请不要因我的悲观性情而轻蔑我。快乐并非不亲近我,只是我心甘清醒的苦楚,谁又能认同这有
        如自虐的作为。我告诉你的,你理解也好,费解也罢,但万望你今后能就此追迹我一切行为所暗蕴的来自
        心灵驱使的蛛丝。

        \hspace*{2em} 你瞧,我又表现地一点也不为人考虑的自说自话,我希望你了解我,却也未明明白白地
        坦露造成我这种特性的缘由。所以,让我作个野生的心理医生,向你阐发对自己“病情”的分析吧。我知道,
        凡后所有事情的成因都能在童年寻到端倪。我的无知岁月可称幸福,在多数时候并未受过父母亲密的管束
        和严格的教导,向来由着性子自然发展,跟着坏孩子便学坏,那时候做过许多违离道德的事,现今依然
        时时让我悔疚。后来令我改变的,来自我学龄时的实际监护人——我的大姨母,同样地,也是我哥哥的监护人。
        约摸在初中时期,我的学习成绩有不小的跌落,她认为是我同小伙伴娱戏过甚而误了学习,而某一次她发现
        我正与一些人做着危险的事,便急忙喝止了我且不让我再做同样的事,那之后,我便再未同他们玩耍了。
        一方面,不能让家人们过多担心,另一方面,我在爱好上面也未能与他们有共识,如此索性就一个人在一小
        方天地内自在。其后,像是“理所应当”的,我又学习好起来了,只我不能觉察内中有何关联,但也惯习了
        这样的自我规束。可幸的是,我十分安适于独自的处境,原因是我渐渐耽迷于不切实的幻想(这在我很小的
        时候便出现了),那所有虚渺的期愿都在专设的梦境中活现,我多么的满足畅怀以致醒来尽是失意,这成了
        我此后常以狎弄的技巧。每置身于不如意的境地,便将自己与现实剥离,灵魂去往自由的国度,任其何种遭遇
        都不能伤损我的核心,而负面地,任何幸福快乐的情状都不会切身领受,因我早已远远的避逃了啊。这是我
        所以为的自身的病灶。

        \hspace*{2em} 因为悲观,造就了我完美主义的偏向,或可能是反向作用。不能尽善都是错,我偏是这样的极端,
        稍有差便不会称意。这般的坏脾性只是朝向自己,对旁人至多不与交往,所以我难以维持人数过多的社交,那会惊人
        的耗费我为存不多的自由精力,我的精神会时时地游离往臆想的美妙世界。怕自己给别人有偏已愿的失望,便从不
        主动予以承诺,知我的人明白我的口是心非,更多的人想是早已掉转。唉,我说过了我是这样矛盾丛生、百纠千结
        的一个人了。所以,我应当一个人,永远地一个人,至少在这乖违的性子得以好转以前,莫能与人产生过深的牵绊,
        那种生活一起的牵扯。

        \hspace*{2em} 好了,我想可以在此结束了,更多话望有机会再言。

}
\end{adjustwidth}
\newpage