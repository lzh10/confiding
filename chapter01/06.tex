\section{八月十五日}
\vspace*{2\ccwd}
\begin{adjustwidth}{0cm}{0cm}
    \Large\ckai{
        王: 

        \hspace*{2em} 很开心能再次与你一谈。这次想说的是一段演绎的故事,故事的原初是一次搭车中的闲谈。
        本是多个月前的经历,而我始终不忘其间我笨拙的应答,其后是我反复的回想、反复的模拟。如今,抛舍真实,
        给你显现一个虚造的情境和对话。

        \hspace*{2em} 缘起是,因同一事到了某处,事毕,朋友的朋友顺道送我们回去。路上,不使车中无言笑以至
        空然静默的氛围,她便聊了起来。谈及了这次来的始末,又问了我们相关的情况,间杂了偶遇不守交通规则的
        路人的事宜,谈及了吃,谈及了工作加班......诸如此些。途中,她说到自己家距公司远,约略半个小时的车程
        (在这不大的城区确属远途了),而后又自嘲到自己路痴,不记得每天上班的路,次次需得导航。大概是同有此感,
        我便说道: 这让我想起了一个民间传说,类同与西方民俗神话里的牙仙子,我们这儿也有一种精怪,专以取食人的
        记忆乃至情感为生。入夜,待人们都沉入睡梦中,它们就静悄悄地潜到你思海的深处,撷取记忆的果子为食。又因为,
        人们最多生产的是那些负面的不开心的情绪,故而,这算主粮了,也是它们最先下口的。反受益于此,多数的人也
        能因一夜好眠而消除了昨昔难堪的回忆。意料外的是,它们中也有一些个食口偏奇的家伙,好似些寻野味的人,
        它们不安于只填肚子,又贪好了旁个不常会吃的,被偷尝了的人就会变得路痴呀,脸盲呀,或是总记不得别人的名字啦,
        有严重的,被饱食了快乐美好的珍贵记忆而后就此悒悒不乐了。所以哪,你只是碰上了比较顽皮的一类食梦仙子,
        只捣蛋不作恶的。
        
        听了我一番胡诌言语,她咯咯地笑了起来:“你哪里听的这个故事啊,还有食梦仙子这个名字,像抄袭来的。”
        
        我解释说:“哈哈,本来也是临时起兴杜撰的。不过照这样说起来,我也常常被它们光顾了。”

        “那都吃了你的什么呀?”她像是有兴致的一问。

        我忽而故作严肃:“快乐。我不快乐了。”不待有所回应,又转而轻松说道:“嗯,除此外,我不仅路痴,也脸盲,
        只见过几面的人记不住脸也记不住名,路上遇上不敢先打招呼。还有好几次对着陌生人先傻笑,临近了想打招呼
        发现认错了人,傻笑也僵住了,真的成了傻子”

        “那你比我严重多了,下次路上遇到你会不会就不认得我了”

        “应该不会的,就算认不得,还是会对着你傻笑的,到时怕你会不认我这个傻子”

        “可能的,路上遇上傻子也会避开走”

        “你没得同情心,伤了傻子的自尊”

        “傻子也会表现自尊吗,那不是装傻嘛”

        “是啊,我也只是像个傻子,不是真的傻子”

        “呵呵,也算的”

        ...... 

        \hspace*{2em} 对话似无有终竟,可事实是路程不会如此长,我们也没有说过如此多的话。车上有旁人,
        我想说的不会说,即使止两人,我也更无法出声了。于此,平白编造了一个梦,无人知晓,只与你分享。
    }
\end{adjustwidth}
\newpage