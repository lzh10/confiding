\section{八月二十六日}
\vspace*{2\ccwd}
\begin{adjustwidth}{0cm}{0cm}
    \Large\ckai{
        王: 

        \hspace*{2em} 近来,因着身体有恙,并带了心情也连受些痛苦。捱迫的思维放大了繁乱,久禁的孤闷屈折了坚韧,
        放由我无凭而惶惑。知其无望,总还希望,频仍失望,不许绝望。

        \hspace*{2em} 像这样的,我总要发端使自己不致快乐的情由,如酒者的常常病酒,而后便可领略、便可剥取,是身心
        轻快时从无以体味的收浓的思酿。饮之,四体无骸,中心沉沉。请一日的静定,放逸思游的空遐,令所有幻愿的场景纷纷
        显见,是梦中戏复戏中梦,总总杂杂心心念念,怎生的不可能都要可能,噫呀呀云胡不可?最好,你也入来,不必请邀,
        不依剧本的预设,随你自好、自决来去。偶或突如其来的招呼使我呆愕的脸满溢笑意,又或霎眼间悄默无息让我
        目寻不着而后怅恍有失。我是未见过你的,可我定能识得你,不凭相貌,不因形姿,一刹那星点、只此灵犀的相惜,万千
        纷纭中你兀然出众,恰我青睐之中。是以,你莫辞拒,请来我梦中。

        \hspace*{2em} 梦想的是,幸福故事里比比见到的终局,邂逅相遇,结成欢好。自此衍开,每一日每一刻,琐末事情、细碎生活,
        尽心思关照,体察变化的感受,欣慰于一丝快乐的回馈,平淡而长久。噢,又岂平淡,日起时晨霞的问候,黄昏后林阴的漫步,
        朗夜中共数的缺圆。日月更接,四时轮迭,任东风吹絮北风扬雪,夏雷乍破秋水流长,白头许约,偕情同守。
        
        \hspace*{2em} 这儿,同居的屋所,百余平。四室,南向二,为主次卧,分别内外,隔客厅;北向二,外小卧,内书室,隔厨房。另有阳台,
        毗邻客厅并次卧而南,长差略十米,宽三尺许。此则全屋之概貌,虽不大,心足矣。一方小小空间,憬憧将来,愿与经历许多多人间事。
        先则,请来一二毛茸茸小主人,或是猫类,或是犬类,如若双全,最是得意。不约以品种,不强以等相,赖缘份之幸临,第一眼见便生欢喜。
        如遇,欣然延至家中。看!原来的一团小毛球,现而铺展,四爪伸张,戳一戳它嫩粉粉的鼻子,摇一摇它细条条的尾巴,逗弄得它睁起惺松迷离的睡眼,
        张嘴嘤嘤唤着,急不耐地需求喂食和安抚,待满足,又施然摊卧。纯然乎天真。上邪!如此可爱何?故而,务请个哟小家伙来。还有,还有,
        阳台置成蔬果园样式,四季气候的变化,看它应时萌芽、舒叶、攀藤、发花、落果,是自然的慢慢诉语,知心者听。午后,在阔叶藤株的遮阴下,
        同坐入双人吊椅,圆几上摆了爱喝的酸甜凉饮,共读一本引人的书,不必说话,共情处,相视脉脉。欢游故事里,自在不知,直到日影飞去,月上梢头,
        我们才从迷沉中步出,灯火已辉煌。 

        \hspace*{2em} 逢着假期,早早起了兴头,去到哪处、如何出行、怎样日程,细细预谋着。一旦成行,期愿所达,乐莫乐兮。观游人文景胜,或慕文人之雅韵,
        或叹奇士之殊行,或讶诧匠人之巧工,或领略异时之俗尚。数日之游程,归来时,自生“古今易矣,旧时风流不再”之“嗟夫”,你笑我痴,劝将我道:“便因闻说而有往观之念,
        今亲身一睹,意趣相得,可不偿愿,况与我同游哉”。我应道:“唯唯。是有此理”。其实呢,我更爱自然山水之光景,眼界开旷,寄情舒怀,胸气与天地通。
        适其春光明媚,草绿茵茵之际,无需行远,只周近之公园,足可消磨。煦日和风,沿湖而行而坐而卧,看游人嬉闹:风筝在天上游,皮球在草面溜,语声嘈嘈,笑声朗朗。
        人间春好,莺飞草长,我静想着恰宜的诗、和你。人亦有言,泰山峻拔,黄山秀绝,皆游山之必观,虽意同却不苟求,山无远名,亦有独到,随缘适性,遇到便好。
        等到秋高之日,登览之佳时,且携轻装,行登临之乐。徐疾拾级,随由心意,体力沛则一气而上,气不足则缓歇,寻大石而倚,饮食休憩,俯眺山壑林,抬望峰头云。
        气力又回复,约算了精神和日头,继而再向顶进,势劲勃发,似有竞意。已而终点在望,我忽作慢行,由你先我而至。于是你胜而骄,我赞而服,反又不甘,共许泰岳日出之约。

        \hspace*{2em} 美梦啊,虚枉弥散、四飘,尽向无际处去,无际处更有无尽的虚空。充不满想望的庞芜,美梦得成亦空虚。我借由你而做了一场捕风的梦,望你见谅。
    }
\end{adjustwidth}
\newpage