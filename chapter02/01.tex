\section{二月二十二}
\vspace*{2\ccwd}
\begin{adjustwidth}{0cm}{0cm}
    \Large\ckai{
        王: 

        \hspace*{2em} 新年将去,晚来的一声祝好,可幸是还能及得说些去年的话,自年前慢慢聚起的暂未消散的话。

        \hspace*{2em} 腊月末,是公司放假前的最后一天,几个旧识与新识的朋友在分别前夕的小聚,本有着淡淡的目的,
        目的在当事者的无意下淡去。算得上,互相都认可为知心的友人,这样的晚上也是不需顾忌的,酒更来为兴致催助,桌上
        大家也都放松下来,倾怀地坦露。说着工作上的种种,是事与人心的纷纷,渐渐地,不免而又谈着了恋爱与婚姻的话题,
        同为单身者的众人,观念上或更有共鸣。席间,是一人说,她至今未愿找对象是想像不到如何的两人能融洽的共处一生,
        更多地是能预见到两个人在性格上的不合、在生活中的龃龉,致使双方在委屈求全中相互绊扯、相互折磨......这些是我
        能记得所闻的大致,其实,此间我也提过了差仿的想法,而近来又更细细想了,便理顺些向你也谈谈:

        \hspace*{2em} 关于亲密关系,我所奢望的两种状态,一曰嵌合(手作抱拳状),即言一者突出,一者内陷,二者都不
        规则也不完全,然遇着却能相合,每每我所给予的都正是你所需要的,是谓我知你心,而你总以恰当的回应让我保有热情,是谓
        你明我意,我们能在情感上交融如一,成为完整的同体;另一曰贴合(掌作合什状),即言二者皆是自立而完全之人,
        分开可独善其身,接触时尺线分明,全心相对而不过分侵入,你并非另一半的我,我们既可在共同的爱好和追求上同调,
        亦可在差别处各自分唱,不求形式上的合一,而为理想和信念同心,以尊重和坦诚、以互爱与理解携手此生。于我的心性而言,
        则更期望于后者的状态。生活中,更多的是没有通情知意的爱,也没有达理明哲的爱,有的是以爱为名义去刺挠、去牵扯、去伤害,
        得到的又怎会是一个令人歆羡的关系呢?又如何愿迫急的自投啊?

        \hspace*{2em} 在此前,想的有更多、更杂乱,却一个有因果相循演进的脉络且丰满的场景,现下是不能够复述的真切,我呀,
        难能将想法落字,更难落行,空有这多许的不称意,许多是要向你展诉的心肠。


}
\end{adjustwidth}
\newpage