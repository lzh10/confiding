\part{因由}

\vspace*{2\ccwd}

\hspace*{2em} 必要分裂两个我了,已是太久久地寂寞无人付心交语。又有谁可相诉,又有谁与应答呢?
自生一个渴想吐白心声的灵魂,收拢着细纤缭绕的烦思,点点理束,有源来而无终竟,都流去向渺茫的大海。
向未名的人写信,无有期求太不可能的回音,如往前无以自答的许多的自问。可除却我,谁更能解我忧惑?
不自期,衍出又一客体的灵魂,接纳本我所有的情感的流,水泊渐积,成河成湖或成海,自甘自苦,自可回溯,
逆涌洄游。感同了所有心相交杂的情态,反观了所有痴嗔爱怨的由来,这颠翻覆涌的内在,原来是我,怎可是我,
必然是我。

\vspace*{2\ccwd}

\hspace*{2em} 近来,读到《少年维特的烦恼》,心善于书信体的写作方式,且无论我是否有笔友这回事,
试将日之所感后之所思写入信中,再向某位慢慢叙来,他必知我甚详,通晓我隐秘于字间的种种无可告人,
所以,可不必顾忌地兴笔畅言,或多胡乱鄙薄之思亦堪展露。又作别想:若是写向自己,更可通明无误,以日记
体写录有何不可?确实,此前记过类同日记的随篇,像是四下漫流的洇水,难寻脉络。许多无来由的思怀,不需
自我解释,就这样的“儵而来兮忽而逝”,不足自审自省。当我需向旁人写明何我所见、何作此想,必当梳理逻辑、
辨析因果,费一番努力将自己看清,才能平白晓畅地达意,脱离本位而旁观,一点点搭建异别于主观自视的样貌。

\vspace*{2\ccwd}

\hspace*{2em} 接下去,会有怎样的路程,不作预想,且行一行吧。
\newpage